% !TEX encoding = UTF-8
% !TEX program = lualatex

% author: essethon
% email: xcalcor@gmail.com
% ----------------------------

\documentclass[a4paper]{scrartcl}

% Tikz & graphics setting
\usepackage{tikz}
\graphicspath{{fig/}}


\usepackage{amssymb}
\usepackage{amsmath,bm,mathrsfs}
% \usepackage{boxedminipage}
% \usepackage{cases}
\usepackage{booktabs}
\usepackage{calc}
\usepackage[font=small]{caption}
  \renewcommand{\figurename}{图}
  \renewcommand{\tablename}{表}
\usepackage{colortbl}
\usepackage{enumitem} % custom enumerate environment
\usepackage{float}
  \usepackage[section]{placeins} % to stop floats at section boundaries
\usepackage{fontspec}
  \setmainfont{Linux Libertine O}[
    BoldFont = Linux Libertine O Bold,
    ItalicFont = Linux Libertine O Italic,
    BoldItalicFont = Linux Libertine O Bold Italic
  ]
  \setsansfont{Linux Biolinum O}[
    BoldFont = Linux Biolinum O Bold,
    ItalicFont = Linux Biolinum O Bold
  ]
  % \setmonofont{Droid Sans Mono for Powerline}
\usepackage{framed}
\usepackage{geometry}
\geometry{left=1.8cm, right=1.8cm, top=2cm, bottom=2.5cm}
\usepackage{graphicx}
\usepackage{hologo}
\usepackage{hyperref}
  \hypersetup{colorlinks=false, pdfborder=0 0 0, pdfencoding=auto}
\usepackage{indentfirst}
\usepackage{latexsym}
\usepackage{listings}
  \lstset{language=Matlab,
    basicstyle=\small\ttfamily, % use mono font
    commentstyle=\color{gray},
    breaklines=true,
    frame=single,
    numbers=left,
    numbersep=5pt,
    numberstyle=\small\color{gray},
    showspaces=false,
    stepnumber=5,
  }
\usepackage{multicol}
\usepackage{multido} % Cycle
\usepackage{pgf}
\usepackage{pstricks}
\usepackage{pst-node}
\usepackage{setspace}
\renewcommand{\baselinestretch}{1.2}
\usepackage{siunitx}
\usepackage{luatexja}
\usepackage{luatexja-fontspec}
  \setmainjfont[BoldFont = FandolSong-Bold]{FandolSong-Regular}
  \setsansjfont[BoldFont = FandolHei-Bold]{FandolHei-Regular}
  % \setmonojfont{WenQuanYi Micro Hei Mono}
  \newjfontface\hei{FandolHei-Regular}
  \newjfontface\Hei{FandolHei-Bold}
  \newjfontface\Song{FandolSong-Bold}
  \newjfontface\fang{FandolFang-Regular}
  \newjfontface\kai{FandolKai-Regular}

\usepackage{luatexja-ruby}

% Indent settings
\setlength\parindent{2em}

% bibstyle
\bibliographystyle{plain}
\usepackage{natbib}
\renewcommand{\refname}{参考文献}

% Title
\title{中文 ARTICLE}
\author{essethon}
\begin{document}
\maketitle
% \begin{multicols}{2}

本文档旨在积累一部分 \LaTeX{} 代码(尤其是 Preamble 部分),以便日后写作业、写文档直接取用。此前(截至 2016 年底),我一直使用 \Hologo{XeLaTeX} 引擎与 \verb|xeCJK| 宏包来解决中文排版问题,因为某些理由,打算从今天起迁移到 \Hologo{LuaLaTeX}. 

 \section{C/J 字体}\label{sec:zhong_wen_zi_ti_}
\Hologo{LuaTeX}-ja 提供了 \texttt{luatexja-fontspec} 宏包,对 \texttt{fontspec} 宏包进行了封装,藉此提供了一些与 \texttt{fontspec} 类似的字体设定命令。如改变局部字体的 \verb|\jfontspec| 命令及本文档在 Preamble 部分用到的一些命令。
\subsection{简体中文}
本文档的默认中文字体和中文无衬线字体分别分别采用了 \verb|FandolSong| 和 \verb|FandolHei| 字体及它们对应的\textbf{粗体}。\verb|Fandol| 系列字体由马起园等人整理开发。对于 \verb|Arch Linux|, 可以从 \verb|AUR| 中安装 \verb|otf-fandol| 包来获得这些字体,该软件包包含了 {\hei{黑体}(及{\Hei{粗体}})}、宋体(及{\Song{粗体}})、{\fang{仿宋}}和{\kai{楷体}},基本能够满足日常中文写作对字体的需求。

\begin{table}[htpb]
  \centering
  \begin{tabular}{lll}
    \toprule
    \multicolumn{1}{c}{\textbf{字体名}} & \multicolumn{1}{c}{\textbf{文件名}} & \multicolumn{1}{c}{\textbf{样例}} \\
    \midrule
    FandolSong-Regular & \texttt{FandolSong-Regular.otf} & {我能吞下玻璃而不伤身体} \\
    FandolSong-Bold & \texttt{FandolSong-Bold.otf} & {\Song{我能吞下玻璃而不伤身体}} \\
    FandolHei-Regular & \texttt{FandolHei-Regular.otf} & {\hei{我能吞下玻璃而不伤身体}} \\
    FandolHei-Bold & \texttt{FandolHei-Bold.otf} & {\Hei{我能吞下玻璃而不伤身体}} \\
    FandolFang-Regular & \texttt{FandolFang-Regular.otf} & {\fang{我能吞下玻璃而不伤身体}}\\
    FandolKai-Regular & \texttt{FandolKai-Regular} & {\kai{我能吞下玻璃而不伤身体}}\\
    \bottomrule
  \end{tabular}
\end{table}

\subsection{日本語}
\LaTeX{} で日本語を書きましょう!
\section{数学环境}\label{sec:shu_xue_huan_jing_}

\section{图表}\label{sec:tu_biao_}

\section{代码引用}\label{sec:dai_ma_yin_yong_}

\section{切换到 \Hologo{LuaTeX} 的理由}\label{sec:qie_huan_dao_luatex_de_li_you_}
  \begin{itemize}
    \item 就是想用 \Hologo{LuaTeX}-ja!
    \item \ltjruby{君|の|名|は}{きみ||な|}
  \end{itemize}
  
  

\end{document}
